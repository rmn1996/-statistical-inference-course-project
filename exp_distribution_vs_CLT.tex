% Options for packages loaded elsewhere
\PassOptionsToPackage{unicode}{hyperref}
\PassOptionsToPackage{hyphens}{url}
%
\documentclass[
]{article}
\usepackage{amsmath,amssymb}
\usepackage{lmodern}
\usepackage{iftex}
\ifPDFTeX
  \usepackage[T1]{fontenc}
  \usepackage[utf8]{inputenc}
  \usepackage{textcomp} % provide euro and other symbols
\else % if luatex or xetex
  \usepackage{unicode-math}
  \defaultfontfeatures{Scale=MatchLowercase}
  \defaultfontfeatures[\rmfamily]{Ligatures=TeX,Scale=1}
\fi
% Use upquote if available, for straight quotes in verbatim environments
\IfFileExists{upquote.sty}{\usepackage{upquote}}{}
\IfFileExists{microtype.sty}{% use microtype if available
  \usepackage[]{microtype}
  \UseMicrotypeSet[protrusion]{basicmath} % disable protrusion for tt fonts
}{}
\makeatletter
\@ifundefined{KOMAClassName}{% if non-KOMA class
  \IfFileExists{parskip.sty}{%
    \usepackage{parskip}
  }{% else
    \setlength{\parindent}{0pt}
    \setlength{\parskip}{6pt plus 2pt minus 1pt}}
}{% if KOMA class
  \KOMAoptions{parskip=half}}
\makeatother
\usepackage{xcolor}
\IfFileExists{xurl.sty}{\usepackage{xurl}}{} % add URL line breaks if available
\IfFileExists{bookmark.sty}{\usepackage{bookmark}}{\usepackage{hyperref}}
\hypersetup{
  pdftitle={Investigate the exponential distribution in R and compare it with the Central Limit Theorem},
  pdfauthor={Raman Agnihotri},
  hidelinks,
  pdfcreator={LaTeX via pandoc}}
\urlstyle{same} % disable monospaced font for URLs
\usepackage[margin=1in]{geometry}
\usepackage{color}
\usepackage{fancyvrb}
\newcommand{\VerbBar}{|}
\newcommand{\VERB}{\Verb[commandchars=\\\{\}]}
\DefineVerbatimEnvironment{Highlighting}{Verbatim}{commandchars=\\\{\}}
% Add ',fontsize=\small' for more characters per line
\usepackage{framed}
\definecolor{shadecolor}{RGB}{248,248,248}
\newenvironment{Shaded}{\begin{snugshade}}{\end{snugshade}}
\newcommand{\AlertTok}[1]{\textcolor[rgb]{0.94,0.16,0.16}{#1}}
\newcommand{\AnnotationTok}[1]{\textcolor[rgb]{0.56,0.35,0.01}{\textbf{\textit{#1}}}}
\newcommand{\AttributeTok}[1]{\textcolor[rgb]{0.77,0.63,0.00}{#1}}
\newcommand{\BaseNTok}[1]{\textcolor[rgb]{0.00,0.00,0.81}{#1}}
\newcommand{\BuiltInTok}[1]{#1}
\newcommand{\CharTok}[1]{\textcolor[rgb]{0.31,0.60,0.02}{#1}}
\newcommand{\CommentTok}[1]{\textcolor[rgb]{0.56,0.35,0.01}{\textit{#1}}}
\newcommand{\CommentVarTok}[1]{\textcolor[rgb]{0.56,0.35,0.01}{\textbf{\textit{#1}}}}
\newcommand{\ConstantTok}[1]{\textcolor[rgb]{0.00,0.00,0.00}{#1}}
\newcommand{\ControlFlowTok}[1]{\textcolor[rgb]{0.13,0.29,0.53}{\textbf{#1}}}
\newcommand{\DataTypeTok}[1]{\textcolor[rgb]{0.13,0.29,0.53}{#1}}
\newcommand{\DecValTok}[1]{\textcolor[rgb]{0.00,0.00,0.81}{#1}}
\newcommand{\DocumentationTok}[1]{\textcolor[rgb]{0.56,0.35,0.01}{\textbf{\textit{#1}}}}
\newcommand{\ErrorTok}[1]{\textcolor[rgb]{0.64,0.00,0.00}{\textbf{#1}}}
\newcommand{\ExtensionTok}[1]{#1}
\newcommand{\FloatTok}[1]{\textcolor[rgb]{0.00,0.00,0.81}{#1}}
\newcommand{\FunctionTok}[1]{\textcolor[rgb]{0.00,0.00,0.00}{#1}}
\newcommand{\ImportTok}[1]{#1}
\newcommand{\InformationTok}[1]{\textcolor[rgb]{0.56,0.35,0.01}{\textbf{\textit{#1}}}}
\newcommand{\KeywordTok}[1]{\textcolor[rgb]{0.13,0.29,0.53}{\textbf{#1}}}
\newcommand{\NormalTok}[1]{#1}
\newcommand{\OperatorTok}[1]{\textcolor[rgb]{0.81,0.36,0.00}{\textbf{#1}}}
\newcommand{\OtherTok}[1]{\textcolor[rgb]{0.56,0.35,0.01}{#1}}
\newcommand{\PreprocessorTok}[1]{\textcolor[rgb]{0.56,0.35,0.01}{\textit{#1}}}
\newcommand{\RegionMarkerTok}[1]{#1}
\newcommand{\SpecialCharTok}[1]{\textcolor[rgb]{0.00,0.00,0.00}{#1}}
\newcommand{\SpecialStringTok}[1]{\textcolor[rgb]{0.31,0.60,0.02}{#1}}
\newcommand{\StringTok}[1]{\textcolor[rgb]{0.31,0.60,0.02}{#1}}
\newcommand{\VariableTok}[1]{\textcolor[rgb]{0.00,0.00,0.00}{#1}}
\newcommand{\VerbatimStringTok}[1]{\textcolor[rgb]{0.31,0.60,0.02}{#1}}
\newcommand{\WarningTok}[1]{\textcolor[rgb]{0.56,0.35,0.01}{\textbf{\textit{#1}}}}
\usepackage{graphicx}
\makeatletter
\def\maxwidth{\ifdim\Gin@nat@width>\linewidth\linewidth\else\Gin@nat@width\fi}
\def\maxheight{\ifdim\Gin@nat@height>\textheight\textheight\else\Gin@nat@height\fi}
\makeatother
% Scale images if necessary, so that they will not overflow the page
% margins by default, and it is still possible to overwrite the defaults
% using explicit options in \includegraphics[width, height, ...]{}
\setkeys{Gin}{width=\maxwidth,height=\maxheight,keepaspectratio}
% Set default figure placement to htbp
\makeatletter
\def\fps@figure{htbp}
\makeatother
\setlength{\emergencystretch}{3em} % prevent overfull lines
\providecommand{\tightlist}{%
  \setlength{\itemsep}{0pt}\setlength{\parskip}{0pt}}
\setcounter{secnumdepth}{-\maxdimen} % remove section numbering
\ifLuaTeX
  \usepackage{selnolig}  % disable illegal ligatures
\fi

\title{Investigate the exponential distribution in R and compare it with
the Central Limit Theorem}
\author{Raman Agnihotri}
\date{18 July, 2022}

\begin{document}
\maketitle

\hypertarget{overview}{%
\subsection{Overview}\label{overview}}

The purpose of this data analysis is to investigate the exponential
distribution and compare it to the Central Limit Theorem. For this
analysis, the lambda will be set to 0.2 for all of the simulations. This
investigation will compare the distribution of averages of 40
exponentials over 1000 simulations.

\hypertarget{simulations}{%
\subsection{Simulations}\label{simulations}}

Set the simulation variables lambda, exponentials, and seed.

\begin{Shaded}
\begin{Highlighting}[]
\NormalTok{ECHO}\OtherTok{=}\ConstantTok{TRUE}
\FunctionTok{set.seed}\NormalTok{(}\DecValTok{1337}\NormalTok{)}
\NormalTok{lambda }\OtherTok{=} \FloatTok{0.2}
\NormalTok{exponentials }\OtherTok{=} \DecValTok{40}
\end{Highlighting}
\end{Shaded}

Run Simulations with variables

\begin{Shaded}
\begin{Highlighting}[]
\NormalTok{simMeans }\OtherTok{=} \ConstantTok{NULL}
\ControlFlowTok{for}\NormalTok{ (i }\ControlFlowTok{in} \DecValTok{1} \SpecialCharTok{:} \DecValTok{1000}\NormalTok{) simMeans }\OtherTok{=} \FunctionTok{c}\NormalTok{(simMeans, }\FunctionTok{mean}\NormalTok{(}\FunctionTok{rexp}\NormalTok{(exponentials, lambda)))}
\end{Highlighting}
\end{Shaded}

\hypertarget{sample-mean-versus-theoretical-mean}{%
\subsection{Sample Mean versus Theoretical
Mean}\label{sample-mean-versus-theoretical-mean}}

\hypertarget{sample-mean}{%
\paragraph{Sample Mean}\label{sample-mean}}

Calculating the mean from the simulations with give the sample mean.

\begin{Shaded}
\begin{Highlighting}[]
\FunctionTok{mean}\NormalTok{(simMeans)}
\end{Highlighting}
\end{Shaded}

\begin{verbatim}
## [1] 5.055995
\end{verbatim}

\hypertarget{theoretical-mean}{%
\paragraph{Theoretical Mean}\label{theoretical-mean}}

The theoretical mean of an exponential distribution is lambda\^{}-1.

\begin{Shaded}
\begin{Highlighting}[]
\NormalTok{lambda}\SpecialCharTok{\^{}{-}}\DecValTok{1}
\end{Highlighting}
\end{Shaded}

\begin{verbatim}
## [1] 5
\end{verbatim}

\hypertarget{comparison}{%
\paragraph{Comparison}\label{comparison}}

There is only a slight difference between the simulations sample mean
and the exponential distribution theoretical mean.

\begin{Shaded}
\begin{Highlighting}[]
\FunctionTok{abs}\NormalTok{(}\FunctionTok{mean}\NormalTok{(simMeans)}\SpecialCharTok{{-}}\NormalTok{lambda}\SpecialCharTok{\^{}{-}}\DecValTok{1}\NormalTok{)}
\end{Highlighting}
\end{Shaded}

\begin{verbatim}
## [1] 0.05599526
\end{verbatim}

\hypertarget{sample-variance-versus-theoretical-variance}{%
\subsection{Sample Variance versus Theoretical
Variance}\label{sample-variance-versus-theoretical-variance}}

\hypertarget{sample-variance}{%
\paragraph{Sample Variance}\label{sample-variance}}

Calculating the variance from the simulation means with give the sample
variance.

\begin{Shaded}
\begin{Highlighting}[]
\FunctionTok{var}\NormalTok{(simMeans)}
\end{Highlighting}
\end{Shaded}

\begin{verbatim}
## [1] 0.6543703
\end{verbatim}

\hypertarget{theoretical-variance}{%
\paragraph{Theoretical Variance}\label{theoretical-variance}}

The theoretical variance of an exponential distribution is (lambda *
sqrt(n))\^{}-2.

\begin{Shaded}
\begin{Highlighting}[]
\NormalTok{(lambda }\SpecialCharTok{*} \FunctionTok{sqrt}\NormalTok{(exponentials))}\SpecialCharTok{\^{}{-}}\DecValTok{2}
\end{Highlighting}
\end{Shaded}

\begin{verbatim}
## [1] 0.625
\end{verbatim}

\hypertarget{comparison-1}{%
\paragraph{Comparison}\label{comparison-1}}

There is only a slight difference between the simulations sample
variance and the exponential distribution theoretical variance.

\begin{Shaded}
\begin{Highlighting}[]
\FunctionTok{abs}\NormalTok{(}\FunctionTok{var}\NormalTok{(simMeans)}\SpecialCharTok{{-}}\NormalTok{(lambda }\SpecialCharTok{*} \FunctionTok{sqrt}\NormalTok{(exponentials))}\SpecialCharTok{\^{}{-}}\DecValTok{2}\NormalTok{)}
\end{Highlighting}
\end{Shaded}

\begin{verbatim}
## [1] 0.0293703
\end{verbatim}

\hypertarget{distribution}{%
\subsection{Distribution}\label{distribution}}

This is a density histogram of the 1000 simulations. There is an overlay
with a normal distribution that has a mean of lambda\^{}-1 and standard
deviation of (lambda*sqrt(n))\^{}-1, the theoretical normal distribution
for the simulations.

\begin{Shaded}
\begin{Highlighting}[]
\FunctionTok{library}\NormalTok{(ggplot2)}
\FunctionTok{ggplot}\NormalTok{(}\FunctionTok{data.frame}\NormalTok{(}\AttributeTok{y=}\NormalTok{simMeans), }\FunctionTok{aes}\NormalTok{(}\AttributeTok{x=}\NormalTok{y)) }\SpecialCharTok{+} 
  \FunctionTok{geom\_histogram}\NormalTok{(}\FunctionTok{aes}\NormalTok{(}\AttributeTok{y=}\NormalTok{..density..), }\AttributeTok{binwidth=}\FloatTok{0.2}\NormalTok{, }\AttributeTok{fill=}\StringTok{"\#0072B2"}\NormalTok{, }
                 \AttributeTok{color=}\StringTok{"black"}\NormalTok{) }\SpecialCharTok{+}
  \FunctionTok{stat\_function}\NormalTok{(}\AttributeTok{fun=}\NormalTok{dnorm, }\AttributeTok{arg=}\FunctionTok{list}\NormalTok{(}\AttributeTok{mean=}\NormalTok{lambda}\SpecialCharTok{\^{}{-}}\DecValTok{1}\NormalTok{, }
                                    \AttributeTok{sd=}\NormalTok{(lambda}\SpecialCharTok{*}\FunctionTok{sqrt}\NormalTok{(exponentials))}\SpecialCharTok{\^{}{-}}\DecValTok{1}\NormalTok{), }
                \AttributeTok{size=}\DecValTok{2}\NormalTok{) }\SpecialCharTok{+}
  \FunctionTok{labs}\NormalTok{(}\AttributeTok{title=}\StringTok{"Plot of the Simulations"}\NormalTok{, }\AttributeTok{x=}\StringTok{"Simulation Mean"}\NormalTok{)}
\end{Highlighting}
\end{Shaded}

\begin{verbatim}
## Warning: Ignoring unknown parameters: arg
\end{verbatim}

\includegraphics{exp_distribution_vs_CLT_files/figure-latex/unnamed-chunk-9-1.pdf}

\end{document}
